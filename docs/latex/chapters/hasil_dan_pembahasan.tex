\begin{center}
  \textbf{BAB IV} \\[0.5em]
  \textbf{HASIL DAN PEMBAHASAN}
\end{center}

\section*{4.1 Kebutuhan Hardware dan Software}
Hardware maupun Software yang digunakan dalam penelitian ini adalah sebagai berikut:

\begin{itemize}
  \item Hardware:
    \begin{itemize}
      \item CPU: M3
      \item RAM: 18 GB
      \item Storage: 512GB SSD
      \item VGA: M3 GPU
      \item Monitor: 15.6 inch Full HD
      \item Mouse: Wireless
      \item Keyboard: Wireless
    \end{itemize}
  \item Software:
    \begin{itemize}
      \item Operating System: MacOS Sequoia 15.3.2
      \item IDE: Vs Code, Cursor
      \item Database: PostgreSQL
      \item Framework: Laravel
      \item Programming Language: PHP, JavaScript, HTML, CSS, SQL
    \end{itemize}
\end{itemize}

\section*{4.2 Implementasi}

Langkah berikutnya setelah melakukan perancangan sistem adalah implementasi sistem dalam pengkodean. Berikut adalah hasil implementasi dari sistem yang telah dirancang.

\subsection*{4.2.1 Tampilan UI Dashboard}